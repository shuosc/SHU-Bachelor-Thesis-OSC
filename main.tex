\documentclass[type=doctor, openany, pifootnote]{shuthesis}
% 选项:
%  type=[master|doctor],            % 必选
%  secret,                          % 可选 (如果论文需要保密, 这一项需要打开)
%  pifootnote,                      % 可选(建议打开)
%  openany|openright,               % 可选 (章首页是右开还是任意开, 默认是右开)
%  nocolor                          % 提交最终版本时请打开此选项
% Use the postscript times font!
\usepackage{algorithm}
\usepackage{algorithmicx}
\usepackage{algpseudocode} % algorithmicx 的伪代码扩展
\usepackage{pdfpages}
\usepackage{shuthesis}
\usepackage{times}


%========设置引用格式========
\usepackage{fontspec}
\usepackage{setspace}
\usepackage{enumitem}
\usepackage{etoolbox}
% 设置参考文献格式
\AtBeginEnvironment{thebibliography}{%
    \zihao{-4} % 设置字体大小为小四号
    \setstretch{1.5} % 设置行距为 23 磅
    \setlength{\parindent}{0pt} % 取消段落缩进
    \setlength{\hangindent}{1.27cm} % 设置悬挂缩进
    \justifying % 设置两端对齐
}
\usepackage{gbt7714}
\bibliographystyle{gbt7714-numerical}
\citestyle{super}                   % 默认引用为角标格式,正文格式为\citestyle{numbers}


%========设置表格式========
\usepackage{ctex}
\usepackage{caption}
\usepackage{array}

% 设置表标题格式
% 设置表标题格式
\captionsetup[table]{
    labelfont={\zihao{-4}}, % 编号字体:黑体小四号粗体
    textfont={\zihao{-4}},   % 表标题数字及字母为 Times New Roman 粗体小四号
    labelsep=space,            % 标签与标题之间的间距
    justification=centering,   % 标题居中
    singlelinecheck=off        % 防止单行标题自动居中
}

% 设置表格内容格式
\newcolumntype{L}{>{\zihao{5}\songti}l} % 左对齐,宋体五号
\newcolumntype{C}{>{\zihao{5}\songti}c} % 居中对齐,宋体五号
\newcolumntype{R}{>{\zihao{5}\songti}r} % 右对齐,宋体五号

%========设置图格式========
% 设置图格式
\captionsetup[figure]{
    labelfont={\zihao{-4}}, % 编号字体:黑体小四号粗体
    textfont={\zihao{-4}},   % 图标题数字及字母为 Times New Roman 粗体小四号
    labelsep=space,            % 标签与标题之间的间距
    justification=centering,   % 标题居中
    singlelinecheck=off        % 防止单行标题自动居中
}

%========设置目录格式========
% 修改目录格式
\makeatletter
\renewcommand*\l@subsection[2]{%
    \@dottedtocline{2}{2.5em}{2.5em}{{\xiaosi #1}}{#2}
    }
\renewcommand*\l@section[2]{%
    \@dottedtocline{1}{1.0em}{2.3em}{{\sihao #1}}{#2}
    }
\renewcommand*\l@chapter[2]{%
    \vspace{5pt}% 增加 10pt 的垂直间距
    \@dottedtocline{0}{0em}{0em}{{\heiti\sihao {#1}}}{{#2}}
    \vspace{5pt}% 增加 10pt 的垂直间距
    }
\makeatother

%========设置各级标题格式========
% 英文正文的字体:
\setmainfont[
    BoldFont=Times New Roman Bold,
    ItalicFont=Times New Roman Italic,
    BoldItalicFont=Times New Roman Bold Italic
]{Times New Roman}
% 英文的 sansfont 主要出现在 section 的标题中 (这里就把 Regular 直接设置为了 TimesNewRoman-Bold):
\setsansfont[
    BoldFont=Times New Roman Bold,
    ItalicFont=Times New Roman Bold Italic,
    BoldItalicFont=Times New Roman Bold Italic
]{Times New Roman Bold}
% 英文的 monofont 主要出现在代码中 (\texttt), 可以根据自己的喜好调整:
\setmonofont[
    BoldFont=Times New Roman Bold,
    ItalicFont=Times New Roman Italic,
    BoldItalicFont=Times New Roman Bold Italic
]{Times New Roman}

\graphicspath{{figures/}}

% 如果出现中文无法显示就请取消注释或者注释
% % 中文正文字体:
% \setCJKmainfont[
%     ExternalLocation=./fonts/,
%     AutoFakeBold=true,
%     ItalicFont=SimKai.ttf,
%     BoldItalicFont=SimKai.ttf
% ]{SimSun.ttf}
% % 中文 sansfont, 主要出现在 section 的标题中:
% \setCJKsansfont[
%     ExternalLocation=./fonts/,
%     AutoFakeBold=true
% ]{SimHei.ttf}


% 下面是论文相关信息的填写:
% 论文题目:
\newcommand{\iTitle}{基于摸鱼链的隐私保护摸鱼学习技术研究}
% 学院:
\newcommand{\iSchool}{摸鱼工程与科学学院}
% 专业:
\newcommand{\iMajor}{摸鱼科学与技术}
% 学号:
\newcommand{\iStudentNumber}{20122012}
% 学生姓名:
\newcommand{\iStudentName}{莫雨}
% 指导老师:
\newcommand{\iSupervisorName}{余墨}
% 起讫时间:
\newcommand{\iThesisTime}{2024 年 12 月 30 日至 2025 年 5 月 23 日}

\begin{document}
% 第 1 页:封面页
% !!! 本页不应该做任何修改,相关信息请在 main.tex 填写

\thispagestyle{empty}

\AddToShipoutPictureBG*{%
\AtPageLowerLeft{\includegraphics[width=\paperwidth,height=\paperheight,keepaspectratio]{figures/cover_background.pdf}}
}
~

\vspace{72pt}

\begin{center}
\chuhao[2]
    本科毕业论文(设计)
\end{center}

\begin{center}
\xiaoer[2]
    UNDERGRADUATE  THESIS (PROJECT)
\end{center}


\begin{table}[h]
\xiaoer[1.5]
\centering
\begin{tabular}{m{4em}<{\centering}m{300pt}<{\centering}}
    \makebox[4em][s]{题\hspace{\fill}目:}  &  {\iTitle} \\ \cline{2-2}
\end{tabular}
\end{table}

\begin{table}[h]
\sanhao[2]
\centering
\begin{tabular}{m{5em}<{\centering}m{300pt}<{\centering}}
    \makebox[5em][s]{学\hspace{\fill}院:}  &  {\iSchool} \\ \cline{2-2}
    \makebox[5em][s]{专\hspace{\fill}业:}  &  {\iMajor} \\ \cline{2-2}
    \makebox[5em][s]{学\hspace{\fill}号:}  &  {\iStudentNumber} \\ \cline{2-2}
    \makebox[5em][s]{学生姓名:}  &  {\iStudentName} \\ \cline{2-2}
    \makebox[5em][s]{指导教师:}  &  {\iSupervisorName} \\ \cline{2-2}
    \makebox[5em][s]{起讫日期:}  &  {\iThesisTime} \\ \cline{2-2}
\end{tabular}
\end{table}

% \begin{table}[]
% \xiaosi[2]
% \begin{tabular}{p{2cm}p{7cm}lll}
% \makebox[5em][s]{姓\hspace{\fill}名:} & {你的名字} & 学号:& {你的学号} &  \\
% \makebox[5em][s]{论文题目:} & \multicolumn{3}{l}{你的论文题目} &  \\
%       &      &         &     &  \\
%       &      &         &     & 
% \end{tabular}
% \end{table}



\vspace{54pt}
\include{contents/declaration}

% 如果需要直接覆盖封面和原创性声明,请将下面一行取消注释,并注释上面两行。
% \includepdf[pages={1,2}]{cover.pdf}

\frontmatter
\newpage

\setcounter{page}{1} % skip the content page

{
% \cleardoublepage% Move to first page of new chapter
\let\clearpage\relax% Don't allow page break
\chapter*{摘\ 要}
}
\addcontentsline{toc}{chapter}{摘\ 要}

这里是中文摘要。

摘要的内容需作者简要介绍本论文的主要内容,主要为本人所完成的工作和创新
点。

\vspace{8mm}
\noindent \textbf{关键词}: \TeX; \LaTeX; Template; Thesis

\newpage
{
% \cleardoublepage% Move to first page of new chapter
\let\clearpage\relax% Don't allow page break
\chapter*{ABSTRACT}
}
\addcontentsline{toc}{chapter}{ABSTRACT}

Abstract in English.

\textbf{The content of the abstract requires the author to briefly introduce the main content of this paper, mainly for my work and innovation.}

\vspace{8mm}

\noindent \textbf{Keywords}: \TeX; \LaTeX; Template; Thesis



{
    \hypersetup{linkcolor=black}
    \tableofcontents
    \thispagestyle{shu@nopagefoot}
}


\mainmatter
\include{contents/chap01}
\include{contents/chap02}
\include{contents/chap03}
\include{contents/chap04}
\include{contents/chap05}
\include{contents/conclusion}
\bibliography{reference/refs}

\backmatter
\begin{appendix}
\input{contents/appendix}
\end{appendix}

\include{contents/acknowledgement}

\includepdf[pages={1}]{back-cover.pdf}

\end{document}
